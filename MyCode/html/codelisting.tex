
% This LaTeX was auto-generated from MATLAB code.
% To make changes, update the MATLAB code and republish this document.

\documentclass{article}
\usepackage{graphicx}
\usepackage{color}

\sloppy
\definecolor{lightgray}{gray}{0.5}
\setlength{\parindent}{0pt}

\begin{document}

    
    
\section*{This is the code listing for the final thesis version}


\subsection*{Contents}

\begin{itemize}
\setlength{\itemsep}{-1ex}
   \item Utilities
   \item General Graph Theory
   \item Fourier Analysis on graphs
   \item Diffusion maps
   \item Figures
\end{itemize}


\subsection*{Utilities}

\begin{par}
In this section we describe all the helper functions required for the simulations.
\end{par} \vspace{1em}
\begin{itemize}
\setlength{\itemsep}{-1ex}
   \item Generating Haar wavelets of given size
\end{itemize}

\begin{verbatim}function [Hr]=generate_haar(N)
% Author: Kamlesh Pawar
% Input :
%     N : size of matrix to be generated, N must be some power of 2.
% Output:
%    Hr : Haar matrix of size NxN\end{verbatim}
    
\begin{verbatim}if (N<2 || (log2(N)-floor(log2(N)))~=0)
    error('The input argument should be of form 2^k');
end\end{verbatim}
    
\begin{verbatim}p=[0 0];
q=[0 1];
n=nextpow2(N);\end{verbatim}
    
\begin{verbatim}for i=1:n-1
    p=[p i*ones(1,2^i)];
    t=1:(2^i);
    q=[q t];
end
Hr=zeros(N,N);
Hr(1,:)=1;
for i=2:N;
    P=p(1,i); Q=q(1,i);
    for j= (N*(Q-1)/(2^P)):(N*((Q-0.5)/(2^P))-1)
        Hr(i,j+1)=2^(P/2);
    end
    for j= (N*((Q-0.5)/(2^P))):(N*(Q/(2^P))-1)
        Hr(i,j+1)=-(2^(P/2));
    end
end
Hr=Hr*(1/sqrt(N));
end\end{verbatim}
    \begin{itemize}
\setlength{\itemsep}{-1ex}
   \item Graph Data Structures
\end{itemize}
\begin{par}
Converting a graph from $\texttt{Graph}$ type to $\texttt{struct}$
\end{par} \vspace{1em}

\begin{verbatim}function [G] = graph2struct(H)
% Takes a graph type and outputs a graph struct to use for
% the gsp box
A = adjacency(H, 'weighted');
N = max(size(A));
G.N = N;
G.W = A;
% Assign a default circular embedding
G.coords=[(cos((0:N-1)*(2*pi)/N))',(sin((0:N-1)*(2*pi)/N))'];
G.plotting.limits=[-1,1,-1,1];
G = gsp_graph_default_parameters(G);\end{verbatim}
    \begin{itemize}
\setlength{\itemsep}{-1ex}
   \item Creating a random graph
\end{itemize}

\begin{verbatim}function A = RandomGraph(N,p)
A = zeros(N,N);
for i = 1:N
    for j = i:N
        A(i,j) = binornd(1,p);
        A(j,i) = A(i,j);
    end
end
A = A - diag(diag(A));\end{verbatim}
    
\begin{verbatim}end\end{verbatim}
    \begin{itemize}
\setlength{\itemsep}{-1ex}
   \item Creating an N-Fan graph
\end{itemize}

\begin{verbatim}function A = nfan(N)
%Return an N fan graph with 2n+1 vertices
A = zeros(2*N+1);
A(1,:) = ones(1,2*N+1); %The first vertex is the center of the fan
A(1,1) = 0;
for i = 1:N
    %i loops through each blade
    A(2*i,1) = 1;
    A(2*i + 1, 1) = 1;
    A(2*i, 2*i + 1) = 1;
    A(2*i + 1, 2*i) = 1;
end\end{verbatim}
    
\begin{verbatim}end\end{verbatim}
    \begin{itemize}
\setlength{\itemsep}{-1ex}
   \item Graph Drawing
\end{itemize}
\begin{par}
Computing the circular embedding of an $N$-vertex graph
\end{par} \vspace{1em}

\begin{verbatim}function E = embedding(N,k)
% N is the number of points
% k is the center of the circle
n = 0:1:N-1;
x = cos(n*(2*pi/N)) + k(1);
y = sin(n*(2*pi/N)) + k(2);
E = [x;y]
end\end{verbatim}
    \begin{itemize}
\setlength{\itemsep}{-1ex}
   \item Subgraphs
\end{itemize}
\begin{par}
Finding the boundary from the index set
\end{par} \vspace{1em}

\begin{verbatim}% This function gives the boundary of a subgraph indexed by sub\end{verbatim}
    
\begin{verbatim}% Arguments:
% 1. G is the ambient graph encoded as a struct
% 2. sub indexes the subset\end{verbatim}
    
\begin{verbatim}function Bindex = boundary(G, sub)\end{verbatim}
    
\begin{verbatim}if ~isstruct(G)
   G = graph2struct(G);
end\end{verbatim}
    
\begin{verbatim}A = G.W; % Computing the adjacency matrix of G
Bindex = zeros(length(sub),length(A)); % Stores the index values of the boundary vertices
for i=1:length(sub)
    for j=1:length(A(1,:))
        if A(sub(i),j) ~= 0 && isempty(sub(sub==j))
            Bindex(i,j) = j;
        end
    end
end\end{verbatim}
    
\begin{verbatim}Bindex = unique(Bindex(:));
Bindex = Bindex(Bindex ~=0);\end{verbatim}
    
\begin{verbatim}end\end{verbatim}
    \begin{par}
Computing subgraphs
\end{par} \vspace{1em}

\begin{verbatim}% This function gives three important subgraphs:
% H -- Subgraph of G induced by sub
% I -- Subgraph of G induced by sub with boundary edges
% J -- Subgraph of G induced by sub union boundary vertices
% deltaS -- the boundary
% G -- the graph
% sub -- selection of the vertices\end{verbatim}
    
\begin{verbatim}% First some mopping up. If G isn't a struct, we convert it to a struct
% object. If it is then we keep it.\end{verbatim}
    
\begin{verbatim}function [H,I,J, deltaS] = subs(G,sub)\end{verbatim}
    
\begin{verbatim}if ~isstruct(G)
   G = graph2struct(G);
end
% A stores the adjacency (or weight) matrix\end{verbatim}
    
\begin{verbatim}A = G.W;\end{verbatim}
    
\begin{verbatim}% find boundary vertices\end{verbatim}
    
\begin{verbatim}deltaS = boundary(G,sub);\end{verbatim}
    
\begin{verbatim}% find induced subgraph\end{verbatim}
    
\begin{verbatim}H = gsp_subgraph(G,sub);\end{verbatim}
    
\begin{verbatim}% graph with subgraph and boundary\end{verbatim}
    
\begin{verbatim}AdjS = zeros(length(sub)+length(deltaS)); % Creating an adjacency matrix
AdjS(1:length(sub), 1:length(sub)) = A(sub,sub);
AdjS(length(sub)+1:end,1:length(sub)) = A(deltaS, sub);
AdjS(1:length(sub), length(sub)+1:end) = (A(deltaS, sub))' ;
I = graph(AdjS);
I = graph2struct(I);\end{verbatim}
    
\begin{verbatim}I = gsp_subgraph(G,[sub';deltaS]);
I.W(end-length(deltaS)+1:end,end-length(deltaS)+1:end) = zeros(length(deltaS));\end{verbatim}
    
\begin{verbatim}% % graph induced by subgraph union boundary\end{verbatim}
    
\begin{verbatim}J = subg(G,[sub';deltaS]);\end{verbatim}
    
\begin{verbatim}% AdjS_deltaS = A([sub reshape(deltaS, [1,length(deltaS)])], [sub reshape(deltaS, [1,length(deltaS)])]);
% % Creating reordered adjacency with subgraph vertices first
% J = graph(AdjS_deltaS);
% J = graph2struct(J);
%\end{verbatim}
    \begin{itemize}
\setlength{\itemsep}{-1ex}
   \item Image Processing
\end{itemize}
\begin{par}
Computing the graph of an image
\end{par} \vspace{1em}

\begin{verbatim}function [G] = im2graph(I)
%takes an image as a matrix
%outputs a graph as a struct with lattice coords\end{verbatim}
    
\begin{verbatim}[n,m] = size(I);
W = zeros(n*m);
sigma_p = max(max(I));
coordinates = zeros(n*m,2);
for i=1:n*m
       row = floor((i-1)/n) + 1;
       column = mod((i-1),m) + 1;
       coordinates(i,2) = 1 - (row/n) + (1/n);
       coordinates(i,1) = (column/m) - (1/m);
end\end{verbatim}
    
\begin{verbatim}for i = 1:n*m
    for j = 1:n*m
           row1 = floor((i-1)/n) + 1; %row position of the ith pixel
           column1 = mod((i-1),m) + 1; %column position of the ith pixel
           row2 = floor((j-1)/n) + 1; %row position of the ith pixel
           column2 = mod((j-1),m) + 1; %column position of the ith pixel
           g_distance = norm(coordinates(i,:)-coordinates(j,:));
           p_distance = I(row1, column1) - I(row2,column2);
           W(i,j) = exp(-((g_distance)^2)/(2))*exp(-((p_distance)^2)/(2*sigma_p));
    end
end
G.W = sparse(W);
G.coords = coordinates;
G.plotting.limits=[0,1,0,1];
G = gsp_graph_default_parameters(G);\end{verbatim}
    \begin{itemize}
\setlength{\itemsep}{-1ex}
   \item Nodal Sets
\end{itemize}
\begin{par}
Computing Nodal edges
\end{par} \vspace{1em}
\begin{verbatim}function E = crossings(A,f)
E = 0;\end{verbatim}
\begin{verbatim}N = max(size(A));
      for i = 1:N
          for j =i:N
              if (f(i)*f(j) \ensuremath{<} 0) \&\& (A(i,j) == 1)
                 E = E+1;
              end
          end
      end
end\end{verbatim}
\begin{par}
Computes G(lambda) from  Proposition 2.2.1
\end{par} \vspace{1em}

\begin{verbatim}function G = lowerbound(A, lambda)\end{verbatim}
    
\begin{verbatim}N = max(size(A));
G = 0 ;
   for i = 1:N
       if sum(A(i,:)) < lambda
           G = G + 1 ;
       end
   end\end{verbatim}
    

\subsection*{General Graph Theory}

\begin{itemize}
\setlength{\itemsep}{-1ex}
   \item Simulations with subgraphs and their boundaries
\end{itemize}

\begin{verbatim}% Compute the graph here. Some sample computations are given below\end{verbatim}
    
\begin{verbatim}% Erdos Renyi random graph\end{verbatim}
    
\begin{verbatim}e = 50;
v = 100;
G := Graph::createRandomGraph(v,e, undirected):\end{verbatim}
    
\begin{verbatim}% Random Graph with a bernoulli distributed adjacency matrix
p=0.05;
N = 100;
A = RandomGraph(N,p);
G = graph(A);
plot(G)\end{verbatim}
    
\begin{verbatim}% Path graphs\end{verbatim}
    
\begin{verbatim}vec =  zeros(1,10);
vec(2) = 1;
Adj = toeplitz(vec);
G = graph(Adj)
plot(G)\end{verbatim}
    
\begin{verbatim}% Cyclic graph\end{verbatim}
    
\begin{verbatim}vec = zeros(1,10);
vec(2) = 1; vec(length(vec)) = 1;
Adj = toeplitz(vec);
G = graph(Adj);
plot(G)\end{verbatim}
    
\begin{verbatim}% Complete graph
weights = [1 -2 3 -4 5 -6 7 -8 9];
vec = ones(1,9);
vec(1) = 0;
Adj = toeplitz(vec);\end{verbatim}
    
\begin{verbatim}% Define the graph subset here in row or column vector form\end{verbatim}
    
\begin{verbatim}sub = randperm(100,20);\end{verbatim}
    
\begin{verbatim}create the three important subgraphs
[S, S_deltaS, S_UdeltaS, deltaS] = subs(G,sub);\end{verbatim}
    
\begin{verbatim}define new colormap
mymap = [1 0 0
         0 0 0
         0 0 1];
colormap(mymap);\end{verbatim}
    
\begin{verbatim}% plotting subgraph with boundary\end{verbatim}
    
\begin{verbatim}colours = zeros(1,length(sub)+length(deltaS));
colours(1:length(sub)) = 1;
colours(length(sub)+1:end) = 2;\end{verbatim}
    
\begin{verbatim}% subroutine for assigning colours to edges\end{verbatim}
    
\begin{verbatim}Edges = table2array(S_deltaS.Edges);
Edges = Edges(:,1:2); % Extract the set of edges denoted by ordered pairs
e_colours = zeros(1,length(Edges)); % Set of edges
for i = 1:length(e_colours)
    if Edges(i,1) <= length(sub) && Edges(i,2) <= length(sub)
        e_colours(i) = 1; % case when the edge is in S
    else
        e_colours(i) = 2;
    end
end\end{verbatim}
    
\begin{verbatim}% end of subroutine\end{verbatim}
    
\begin{verbatim}p = plot(S_deltaS, 'MarkerSize', 10, 'LineWidth', 2);
legend('Subgraph', 'Boundary');
p.NodeCData = colours;
p.NodeLabel = [];
p.EdgeCData = e_colours;
p\end{verbatim}
    
\begin{verbatim}% plotting graph induced by subgraph union boundary\end{verbatim}
    
\begin{verbatim}v_colours = zeros(1,length(sub)+length(deltaS));
v_colours(1:length(sub)) = 2;
v_colours(length(sub)+1:end) = 3;\end{verbatim}
    
\begin{verbatim}% % subroutine for assigning colours to edges\end{verbatim}
    
\begin{verbatim}Edges = table2array(S_UdeltaS.Edges);
Edges = Edges(:,1:2); % Extract the set of edges denoted by ordered pairs
e_colours = zeros(1,length(Edges)); % Set of edges
for i = 1:length(e_colours)
    if Edges(i,1) <= length(sub) && Edges(i,2) <= length(sub)
        e_colours(i) = 2; % case when the edge is in S
    elseif Edges(i,1) > length(sub) && Edges(i,1) > length(sub)
        e_colours(i) = 3; % case when the edge is not in S or the boundary
    else
        e_colours(i) = 1;\end{verbatim}
    
\begin{verbatim}    end
end\end{verbatim}
    
\begin{verbatim}% % end of subroutine\end{verbatim}
    
\begin{verbatim}subgraph_location = embedding(length(sub),[0 0]);
boundary_location = embedding(length(deltaS),[4 0]);
location = [subgraph_location boundary_location];
p = plot(S_UdeltaS, 'MarkerSize', 10, 'LineWidth', 2);
legend('Subgraph', 'Boundary');
p.NodeCData = v_colours;
p.NodeLabel = [];
p.EdgeCData = e_colours;
p.XData = location(1,:);
p.YData = location(2,:);
p\end{verbatim}
    

\subsection*{Fourier Analysis on graphs}

\begin{itemize}
\setlength{\itemsep}{-1ex}
   \item Computing Neumann and Dirichlet Operators
\end{itemize}

\begin{verbatim}function [N,D, B, deltaT_S, T_S, N_mat] = Neumann_Dirichlet(G,sub)
% Computes the Neumann and dirichlet operators
% operator on the subgraph induced by sub
% G is the graph. Could be in struct type.
% sub is the selection of vertices\end{verbatim}
    
\begin{verbatim}if ~isstruct(G)
   G = graph2struct(G);
end\end{verbatim}
    
\begin{verbatim}[~, S_deltaS, S_UdeltaS, ~] = subs(G,sub);
L = S_UdeltaS.L; % note that L is stored as sparse double
D = L(1:length(sub), 1:length(sub)); % the dirichlet matrix
B = -L(length(sub)+1:end,1:length(sub)); %the boundary map
l = S_deltaS.L;
deltaT_S = l(length(sub)+1:end, length(sub)+1:end);
N = sparse(D - (B')*(diag(1./diag(deltaT_S)))*B);
diagonal = diag(l);
T_S = diag(diagonal(1:length(sub)));
N_mat = vertcat(sparse(eye(length(sub))), sparse((diag(1./diag(deltaT_S)))*B));\end{verbatim}
    
\begin{verbatim}end\end{verbatim}
    \begin{itemize}
\setlength{\itemsep}{-1ex}
   \item Combinatorial Laplacian Spectra of some common graphs
\end{itemize}

\begin{verbatim}N = 10;
A = RandomGraph(N);
G = graph(A);
plot(G, 'NodeColor', 'r', 'MarkerSize', 10, 'LineWidth', 2);\end{verbatim}
    
\begin{verbatim}% Path graphs\end{verbatim}
    
\begin{verbatim}% vec =  zeros(1,10);
% vec(2) = 1;
% Adj = toeplitz(vec);
% G = graph(Adj)
% plot(G)\end{verbatim}
    
\begin{verbatim}% Cyclic graph
% S
% vec = zeros(1,10);
% vec(2) = 1; vec(length(vec)) = 1;
% Adj = toeplitz(vec);
% G = graph(Adj);
% plot(G)\end{verbatim}
    
\begin{verbatim}% Complete graph
%weights = [1 -2 3 -4 5 -6 7 -8 9];
% vec = ones(1,9);
% vec(1) = 0;
% Adj = toeplitz(vec);\end{verbatim}
    
\begin{verbatim}% Emb = embedding(N);
% G = graph(A);
% L = full(laplacian(G));
% [V, D] = eig(L);
%plot(G, 'XData', Emb(1,:), 'YData', Emb(2,:), 'ZData', V(:,6));
%stem(1:N, V(:,9))\end{verbatim}
    \begin{itemize}
\setlength{\itemsep}{-1ex}
   \item Exploring $|\mathcal{E}|$ vs $G(\lambda)$
\end{itemize}

\begin{verbatim}N = 20;
figure
title('Comparing $$\mathcal{E}$$ to G');
for k = 1:9
    A = RandomGraph(N);
    G = graph(A);
    L = full(laplacian(G));
    [V, D] = eig(L);
    %eigs stores the eigenvalues
    eigs = diag(D);
    %C stores the edge crossing number
    C = zeros(1,N);
    for i = 1:N
        C(i) = crossings(A, V(:,i));
    end
    %Creating the G vector. For some reason Matlab isn't allowing me to broad
    %cast so I'll just do it via a for loop. Sigh.
    G = zeros(1,N);
    for i = 1:N
        G(i) = lowerbound(A, eigs(i));
    end
    subplot(3,3,k)
    plot(eigs(2:N), C(2:N), 'r.');
    hold on
    plot(eigs(2:N), G(2:N)/2, 'b-');
end\end{verbatim}
    \begin{itemize}
\setlength{\itemsep}{-1ex}
   \item Laplacian Spectra of Fans
\end{itemize}

\begin{verbatim}A = nfan(3);
G = graph(A);
D = diag(A*ones(max(size(A)),1));
plot(G);
L = full(laplacian(G));
L_norm = D^(-1/2)*L*D^(-1/2);
[V,D] = eig(L_norm);\end{verbatim}
    

\subsection*{Diffusion maps}

\begin{itemize}
\setlength{\itemsep}{-1ex}
   \item Computing a Laplacian Eigenmap
\end{itemize}

\begin{verbatim}function [Diff_maps] = My_Eigenmaps(G,t,dim)
% Spits out 5 dimensional laplacian eigenmaps given a graph G in struct
% form
N = max(size(G.W));
% compute the random walk matrix here
% t is the scaling factor in diffusion map
lap = gsp_create_laplacian(G, 'normalized'); %lap will be a struct
D = diag(full(lap.W)*(ones(N,1))); %Stores diagonal matrix
M = eye(N) - full(lap.L); %Regularized random walk
[X, Lambda] = eigs(M,dim+1,'largestabs'); %Compute spectral decomposition up to 5 eigenvectors
Phi = D^(1/2)*X; %Phi matrix
Psi = D^(-1/2)*X; %Psi matrix; you want to extract its columns!
Diff_maps = (Psi)*(Lambda^t); %Multiplying each column with the respective eigenvalue
Diff_maps = Diff_maps(:,2:end); %Dropping the first column as it's all a constant
% Each column of diff_maps contains a coordinate of the diffusion map
% to plot it, plot Diff_maps(:,j) in the jth coordinate
end\end{verbatim}
    \begin{itemize}
\setlength{\itemsep}{-1ex}
   \item Laplacian eigenmaps with digital weights
\end{itemize}

\begin{verbatim}First we provide a graph G. Enter graph here
N = 128; % Number of vertices
G = gsp_spiral(N,3); % Creating a graph in struct version\end{verbatim}
    
\begin{verbatim}Diff_maps = My_Eigenmaps(G,1);\end{verbatim}
    
\begin{verbatim}% Plotting the jdim diffusion map
%plot(Diff_maps(:,1),Diff_maps(:,2),'ro');
plot3(Diff_maps(:,3),Diff_maps(:,4),Diff_maps(:,1),'ro');\end{verbatim}
    \begin{itemize}
\setlength{\itemsep}{-1ex}
   \item Diffusion maps for ring, spiral, swiss roll, sphere, and stochastic block graphs
\end{itemize}

\begin{verbatim}% Uncomment each section to visualize the respective laplacian eigenmap\end{verbatim}
    
\begin{verbatim}% Map for rings\end{verbatim}
    
\begin{verbatim}G = gsp_ring(1000);
W = full(G.W);
[mappedX, mapping, lambda] = lapbasic(W, 3, 1, 'JDQR');
for i=0:0.1:1
    plot(mappedX(:,1),mappedX(:,3))
    hold on
end\end{verbatim}
    
\begin{verbatim}% % Map for spiral
%
% G = gsp_spiral(100,3);
% W = full(G.W);
% [mappedX, mapping, lambda] = lapbasic(W, 3, 1, 'JDQR');
% for i=0:0.1:1
%      plot3(((lambda(1))^(-i))*mappedX(:,1),((lambda(2))^(-i))*mappedX(:,2),((lambda(3))^(-i))*mappedX(:,3))
%      hold on
% end\end{verbatim}
    
\begin{verbatim}% % Map for sphere
%
% G = gsp_sphere(100);
% W = full(G.W);
% [mappedX, mapping, lambda] = lapbasic(W, 3, 1, 'JDQR');
% for i=0:0.1:1
%     plot3(((lambda(1))^(-i))*mappedX(:,1),((lambda(2))^(-i))*mappedX(:,2),((lambda(3))^(-i))*mappedX(:,3))
%     hold on
% end\end{verbatim}
    
\begin{verbatim}% Map for swiss roll\end{verbatim}
    
\begin{verbatim}% G = gsp_swiss_roll(500);
% W = full(G.W);
% [mappedX, mapping, lambda] = lapbasic(W, 3, 1, 'JDQR');
% % for i=0:0.1:1
% %     plot(((lambda(1))^(-i))*mappedX(:,1),((lambda(2))^(-i))*mappedX(:,2))
% %     hold on
% % end
% plot3(((lambda(1))^(-i))*mappedX(:,1),((lambda(2))^(-i))*mappedX(:,2),((lambda(3))^(-i))*mappedX(:,3),'o')\end{verbatim}
    
\begin{verbatim}% Map for stochastic block graphs\end{verbatim}
    
\begin{verbatim}% G = gsp_stochastic_block_graph(1024,10);
% W = full(G.W);
% [mappedX, mapping, lambda] = lapbasic(W, 3, 1, 'JDQR');
% for i=0:0.1:1
%     plot3(((lambda(1))^(-i))*mappedX(:,1),((lambda(2))^(-i))*mappedX(:,2),((lambda(3))^(-i))*mappedX(:,3),'o')
%     hold on
% end\end{verbatim}
    

\subsection*{Figures}

\begin{par}
Chapter 1
\end{par} \vspace{1em}
\begin{itemize}
\setlength{\itemsep}{-1ex}
   \item Plotting the haar functions
\end{itemize}

\begin{verbatim}H = generate_haar(512);
times = linspace(0,1,512);
figure;
for i=1:8
    if i==1
        titlestring = strcat("$\varphi$");
        minlim = -1;
        maxlim = 1;
    else
         j = floor(log2(i-1));
         k = (i-1)-2^j;
         titlestring = strcat("$\psi_{",num2str(j),",",num2str(k),"}$");
         maxlim = max(H(i,:));
         minlim = min(H(i,:));
    end
    subplot(2,4,i);
    set(gca,'TickLabelInterpreter','latex');
    set(groot, 'DefaultLegendInterpreter','latex');
    plot(times,H(i,:),'LineWidth',2,'DisplayName','Level 8 Haar approximation');
    yticks([]);
    xticks([0 0.25 0.50 0.75 1]);
    xticklabels({'0', '$\frac{1}{4}$', '$\frac{1}{2}$', '$\frac{3}{4}$', '1'});
    set(gca,'FontSize',16);
    legend('FontSize',11);
    title(titlestring,'interpreter','latex','FontSize',20);\end{verbatim}
    
\begin{verbatim}end\end{verbatim}
    \begin{itemize}
\setlength{\itemsep}{-1ex}
   \item Compressing Heart rate data
\end{itemize}

\begin{verbatim}load BabyECGData;
% figure;
% p1 = plot(times,HR,'-');
% xlabel('Hours');
% ylabel('Heart Rate');
% p1.Color(4) = 0.25;
% hold on;
[a,d] = haart(HR,'integer');
% HaarHR = ihaart(a,d,1,'integer');
% plot(times,HaarHR,'Linewidth',1)
% title('Haar Approximation of Heart Rate')
imz = zeros(10,2048);\end{verbatim}
    
\begin{verbatim}for i = 1:10
    HaarHR = ihaart(a,d,i,'integer');
    imz(i,:) = HaarHR';
end
figure;
subplot(2,1,1);
set(gca,'TickLabelInterpreter','latex');
set(groot, 'DefaultLegendInterpreter','latex');
p1 = plot(times,HR,'-','DisplayName', 'Heart Rate');
xlabel('Hours', 'interpreter','latex', 'FontSize',16);
ylabel('Heart Rate','interpreter', 'latex','FontSize',16);
p1.Color(4) = 0.2;
hold on
p2 = plot(times,imz(3,:),'r-','LineWidth',1,'DisplayName','Level 8 Haar approximation');
hold on
p3 = plot(times,imz(7,:),'k--','LineWidth',2,'DisplayName','Level 4 Haar approximation');
legend('FontSize',11);\end{verbatim}
    
\begin{verbatim}subplot(2,1,2);
colormap copper
set(gca,'TickLabelInterpreter','latex');
set(groot, 'DefaultLegendInterpreter','latex');
image(imz,'CDataMapping', 'scaled');
cbh = colorbar;
cbh.Ticks = [];
ylabel(cbh, 'Heart Rate','interpreter','latex','FontSize',16);
xticks([]);
%xlabel("Hours", 'interpreter','latex','FontSize', 16);
ylabel("Scale (j)", 'interpreter','latex','FontSize', 16);\end{verbatim}
    \begin{itemize}
\setlength{\itemsep}{-1ex}
   \item Pixellating the mandrill
\end{itemize}

\begin{verbatim}load mandrill
%im = imread('Rcirc.png');
im = imresize(X,[512 512]);
%im = im(:,:,1);\end{verbatim}
    
\begin{verbatim}[a,h,v,d] = haart2(im,'integer');
figure;\end{verbatim}
    
\begin{verbatim}for i=0:8
    row = floor(i/3) + 1;
    column = mod(i,3) + 1;
    subplot(3,3,i+1);
    imrec = ihaart2(a,h,v,d,i,'integer');
    colormap parula
    imagesc(imrec);
    title(strcat('Level', " ", num2str(8-i + 1)), 'interpreter', 'latex','FontSize',20);
    axis off;
end\end{verbatim}
    
\begin{verbatim}% to extract a 2^N x 2^N sized image, just pick d(N:end) and run ihaart\end{verbatim}
    \begin{par}
Chapter 2
\end{par} \vspace{1em}
\begin{itemize}
\setlength{\itemsep}{-1ex}
   \item Comparing the DHT and DFT of a 32-bit image with a central spot
\end{itemize}

\begin{verbatim}Img = zeros(32,32);
Img(16,16) = 1;
Img(16,17) = 1;
Img(17,16) = 1;
Img(17,17) = 1;\end{verbatim}
    
\begin{verbatim}FT = fft2(Img);
[a,h,v,d] = haart2(Img, 'integer');\end{verbatim}
    
\begin{verbatim}HT = cell2mat(d(1));\end{verbatim}
    
\begin{verbatim}figure();\end{verbatim}
    
\begin{verbatim}subplot(2,2,1);
imagesc(Img);
caxis('manual');
caxis([-1 1])
title('Original Image','interpreter','latex','FontSize',20);
set(gca,'XColor', 'none','YColor','none')\end{verbatim}
    
\begin{verbatim}subplot(2,2,2);
imagesc(HT);
caxis('manual');
caxis([-1 1])
title('16-point Haar Coefficients','interpreter','latex','FontSize',20);
set(gca,'XColor', 'none','YColor','none')\end{verbatim}
    
\begin{verbatim}subplot(2,2,3);
imagesc(real(FT));
caxis('manual');
caxis([-1 1])
title('Real part of Fourier coefficients','interpreter','latex','FontSize',20);
set(gca,'XColor', 'none','YColor','none')\end{verbatim}
    
\begin{verbatim}subplot(2,2,4);
imagesc(imag(FT));
caxis('manual');
caxis([-1 1])
title('Imaginary part of Fourier coefficients','interpreter','latex','FontSize',20);
set(gca,'XColor', 'none','YColor','none');
cbh = colorbar;
cbh.Ticks = [-1 1];
ylabel(cbh, 'Luminescence/Coefficient Value','interpreter','latex','FontSize',20);\end{verbatim}
    \begin{itemize}
\setlength{\itemsep}{-1ex}
   \item Facial recognition example
\end{itemize}

\begin{verbatim}im = imread('obama.jpg');
im = imresize(im, [512 512]);
[a,h,v,d] = haart2(im,'integer');
D = d(5:end);
H = h(5:end);
V = v(5:end);
Imz = ihaart2(a,H,V,D,1,'integer');
Imz = double(Imz(:,:,1));
G  = im2graph(Imz);
S = gsp_compute_fourier_basis(G);
U = full(S.U);
fiedler_vector = U(:,2);
M = median(fiedler_vector);
classifier = fiedler_vector > 0;
gsp_plot_signal(G,classifier);
colormap flag\end{verbatim}
    \begin{par}
Chapter 3
\end{par} \vspace{1em}
\begin{itemize}
\setlength{\itemsep}{-1ex}
   \item Drawing a circulant matrix
\end{itemize}

\begin{verbatim}Let's create a circulant matrix!\end{verbatim}
    
\begin{verbatim}i = 10;
v = [0 ones(1,i) zeros(1,99-(2*i)) ones(1,i)];\end{verbatim}
    
\begin{verbatim}A = toeplitz([v(1) fliplr(v(2:end))], v);\end{verbatim}
    
\begin{verbatim}P = eye(100);
P = P(randperm(100),:);\end{verbatim}
    
\begin{verbatim}Adj = P*A*P';\end{verbatim}
    
\begin{verbatim}G = graph(Adj);\end{verbatim}
    
\begin{verbatim}G = graph2struct(G);\end{verbatim}
    
\begin{verbatim}Diff_Maps = My_Eigenmaps(G,1,3);\end{verbatim}
    
\begin{verbatim}% Plotting the jdim diffusion map
subplot(1,2,1)
colormap gray
imagesc(Adj);
title("Adjacency Matrix of a circulant graph $V=100$, $k=20$",'fontsize',16,'interpreter','latex');\end{verbatim}
    
\begin{verbatim}subplot(1,2,2);
scatter(Diff_Maps(:,1),Diff_Maps(:,2));
title("2 Dimensional Diffusion embedding",'fontsize',16,'interpreter','latex');
%plot3(Diff_maps(:,1),Diff_maps(:,2),Diff_maps(:,3),'ro');\end{verbatim}
    \begin{itemize}
\setlength{\itemsep}{-1ex}
   \item Embedding a torus
\end{itemize}

\begin{verbatim}G = gsp_torus(32,32);\end{verbatim}
    
\begin{verbatim}Diff_Maps = My_Eigenmaps(G,1,5);\end{verbatim}
    
\begin{verbatim}subplot(1,2,1)
plot3(G.coords(:,1),G.coords(:,2),G.coords(:,3),'ro');
title("Uniformly Sampled points on a Torus",'fontsize',18,'interpreter','latex');\end{verbatim}
    
\begin{verbatim}subplot(1,2,2);
plot3(Diff_Maps(:,1),Diff_Maps(:,2),Diff_Maps(:,3),'bo-');
title("3 Dimensional Diffusion embedding",'fontsize',18,'interpreter','latex');
%plot3(Diff_maps(:,1),Diff_maps(:,2),Diff_maps(:,3),'ro');\end{verbatim}
    \begin{itemize}
\setlength{\itemsep}{-1ex}
   \item Redrawing the circle on a plain background
\end{itemize}

\begin{verbatim}% Extracting the image
Iz = imread('Rcirc.png');
im = imresize(Iz,[32 32]);
im = im(:,:,1);
H = im2graph(im2double(im));
Diff_maps = My_Eigenmaps(H,1,5);
figure;
subplot(2,2,1);
imagesc(Iz(:,:,1));
title('Original Image','interpreter','latex','FontSize',20);
set(gca,'xtick',[]);
set(gca,'ytick',[]);\end{verbatim}
    \begin{itemize}
\setlength{\itemsep}{-1ex}
   \item Exploring the effects of scale
\end{itemize}

\begin{verbatim}Iz = imread('Rcirc.png');
im = imresize(Iz,[32 32]);
im = im(:,:,1);
H = im2graph(im2double(im));
t = [0 0.25 0.5 0.75 1];
 for i=t
     Diff_maps = My_Eigenmaps(H,i,5);
     scale = num2str(i);
     labelstring = strcat('t=',scale);
     scatter3(Diff_maps(:,1),Diff_maps(:,2),Diff_maps(:,3),'DisplayName',labelstring);
     hold on;
 end\end{verbatim}
    \begin{verbatim}legend;\end{verbatim}
\begin{verbatim}scatter(Diff\_maps(:,1),Diff\_maps(:,2),pointsize,Diff\_maps(:,3))
pointsize = 20; colorbar jet;
scatter(Diff\_maps(:,1),Diff\_maps(:,2),pointsize,Diff\_maps(:,3)); colormap jet;\end{verbatim}
\begin{itemize}
\setlength{\itemsep}{-1ex}
   \item Comparing Neumann and Diffusion embeddings of a path as a subgraph of a cycle
\end{itemize}

\begin{verbatim}N = 256;
G = gsp_ring(N);\end{verbatim}
    
\begin{verbatim}sub = N/4:(3*N/4);\end{verbatim}
    
\begin{verbatim}NDiff_Maps = Neumann_DiffMaps(G,sub,3,0.1);\end{verbatim}
    
\begin{verbatim}[S, S_deltaS, S_UdeltaS, deltaS] = subs(G,sub);\end{verbatim}
    
\begin{verbatim}Diff_Maps = gsp_laplacian_eigenmaps(S,3);\end{verbatim}
    
\begin{verbatim}figure;
subplot(2,2,1);
gsp_plot_graph(G);
title('Original Graph','interpreter','latex','FontSize',16);\end{verbatim}
    
\begin{verbatim}subplot(2,2,2);
gsp_plot_graph(S);
title('Subgraph','interpreter','latex','FontSize',16);\end{verbatim}
    
\begin{verbatim}subplot(2,2,3);
gsp_plot_graph(S_deltaS);
title('Subgraph with Boundary','interpreter','latex','FontSize',16);\end{verbatim}
    
\begin{verbatim}subplot(2,2,4);
plot3(NDiff_Maps(:,3),NDiff_Maps(:,2),NDiff_Maps(:,1),'ro','DisplayName','Neumann');
hold on;
plot3(Diff_Maps(:,3),Diff_Maps(:,2),Diff_Maps(:,1),'bo','DisplayName','Diffusion');
title('Diffusion Embeddings','interpreter','latex','FontSize',16);
legend;\end{verbatim}
    \begin{itemize}
\setlength{\itemsep}{-1ex}
   \item Comparing Neumann and Diffusion embeddings of a 2-spiral as a subraph of a 3-spiral
\end{itemize}

\begin{verbatim}N = 256;
G = gsp_spiral(N);\end{verbatim}
    
\begin{verbatim}sub = 1:floor(N/3);\end{verbatim}
    
\begin{verbatim}NDiff_Maps = Neumann_DiffMaps(G,sub,3,0.1);\end{verbatim}
    
\begin{verbatim}[S, S_deltaS, S_UdeltaS, deltaS] = subs(G,sub);\end{verbatim}
    
\begin{verbatim}Diff_Maps = gsp_laplacian_eigenmaps(S,3);\end{verbatim}
    
\begin{verbatim}figure;
subplot(2,2,1);
gsp_plot_graph(G);
title('Original Graph','interpreter','latex','FontSize',16);\end{verbatim}
    
\begin{verbatim}subplot(2,2,2);
gsp_plot_graph(S);
title('Subgraph','interpreter','latex','FontSize',16);\end{verbatim}
    
\begin{verbatim}subplot(2,2,3);
gsp_plot_graph(S_deltaS);
title('Subgraph with Boundary','interpreter','latex','FontSize',16);\end{verbatim}
    
\begin{verbatim}subplot(2,2,4);
plot3(NDiff_Maps(:,3),NDiff_Maps(:,2),NDiff_Maps(:,1),'ro','DisplayName','Neumann');
hold on;
plot3(Diff_Maps(:,3),Diff_Maps(:,2),Diff_Maps(:,1),'bo','DisplayName','Diffusion');
title('Diffusion Embeddings','interpreter','latex','FontSize',16);
legend;\end{verbatim}
    \begin{itemize}
\setlength{\itemsep}{-1ex}
   \item Comparing the Neumann and Diffusion embeddings of a subset of a pointcloud from a sphere sampled from the polar cap as
\end{itemize}

\begin{verbatim}% Experiments with spheres
size = 512;
sphere_graph = gsp_sphere(size);
coordinates = sphere_graph.coords;
elevation = coordinates(:,3);
polarcap = coordinates(elevation > 1/2,:);
%plot3(polarcap(:,1),polarcap(:,2),polarcap(:,3),'ro');
cap = find(elevation > 1/2);
distances = gsp_distanz(coordinates',coordinates');
eps=0.5;
weightmatrix = exp(-(1/(2*(eps)^2))*(distances.^2)) - eye(size);
S = graph(weightmatrix,'upper');
S = graph2struct(S);
S.coords = coordinates;\end{verbatim}
    
\begin{verbatim}% Run Neumann Diffusion on the polar cap
NDiff_Maps = Neumann_DiffMaps(S,cap',5,1);\end{verbatim}
    
\begin{verbatim}% Run Standard Diffusion on the polar cap
[T, T_deltaT, T_UdeltaT, deltaT] = subs(S,cap');
Diff_maps = My_Eigenmaps(T,1,5);\end{verbatim}
    
\begin{verbatim}%plot both\end{verbatim}
    
\begin{verbatim}figure;
subplot(2,2,1);
plot3(coordinates(:,1),coordinates(:,2),coordinates(:,3),'bo','DisplayName','Sphere');
hold on;
plot3(coordinates(cap,1), coordinates(cap,2), coordinates(cap,3),'ro','DisplayName','Cap');
title("Sphere and Polar Cap",'interpreter','latex','FontSize',16);
legend;\end{verbatim}
    
\begin{verbatim}subplot(2,2,2);
plot3(Diff_Maps(:,3),Diff_Maps(:,2),Diff_Maps(:,1),'ro');
title('Diffusion Map','interpreter','latex','FontSize',16);\end{verbatim}
    
\begin{verbatim}subplot(2,2,3);
plot(NDiff_Maps(:,1),NDiff_Maps(:,2),'bo');
title('2-D Neumann Map','interpreter','latex','FontSize',16);\end{verbatim}
    
\begin{verbatim}subplot(2,2,4);
plot3(NDiff_Maps(:,1),NDiff_Maps(:,2),NDiff_Maps(:,3),'bo');
title('3-D Neumann Map','interpreter','latex','FontSize',16);\end{verbatim}
    


\end{document}
    
